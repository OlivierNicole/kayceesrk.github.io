\documentclass[10pt]{article}
\usepackage{charter}
% \usepackage{sourcecodepro-type1-autoinst}
% \usepackage{MinionPro}
\usepackage[usenames,dvipsnames,svgnames]{xcolor}
\usepackage{hyperref}
\usepackage[flushleft,neverdecrease,neveradjust]{paralist}
\usepackage{multicol}
\usepackage{tikz}
\usepackage{pifont}
\pagestyle{empty}
\usepackage{mdwlist}

\usepackage[left=.5cm,top=2cm,right=.5cm,bottom=2.5cm,nohead,nofoot]{geometry}

\definecolor{darkgrey}{HTML}{333333}
\definecolor{grey}{HTML}{4D4D4D}
\definecolor{lightgrey}{HTML}{999999}
\definecolor{green}{HTML}{C2E15F}
\definecolor{orange}{HTML}{FDA333}
% \definecolor{purple}{HTML}{D3A4F9}
\definecolor{purple}{HTML}{9B59B6}
\definecolor{red}{HTML}{FB4485}
\definecolor{blue}{HTML}{6CE0F1}
\definecolor{zest}{HTML}{E67E22}
\linespread{1}
\renewcommand{\labelitemi}{}
\renewcommand{\labelitemii}{}

\newcommand{\lbar}[1]{{\color{#1}\ding{118}}\hspace*{2pt}}

\hypersetup{
  colorlinks=true,
  urlcolor=Sepia,
  linkcolor=RoyalBlue,
  pdfborder= 0 0 0,
  bookmarks=false,
  pdftitle={KC Sivaramakrishnan - CV},
  pdfauthor={KC Sivaramakrishnan},
  pdfsubject={Curriculum Vitae},
  pdfkeywords={KC Sivaramakrishnan, resume, cv, mathematics, cs,
		computer science, phd, programming languages, concurrency, eventual
		consistency, parallelism, multicore, distributed systems,
		non-cache-coherence, cache coherence, NUMA, language runtimes, functional
		programming, Haskell, OCaml, Standard ML, MLton, MultiMLton, PLDI, ICFP,
		ISMM, JFP, weak consistency, relaxed memory, axiomatic semantics,
		operational semantics, cassandra, NoSQL, ZeroMQ, Garbage Collection
}}

% http://tex.stackexchange.com/questions/110236/is-there-a-way-to-get-reverse-numbering-on-the-enumerate-environment
\newenvironment{benumerate}[2]{
    \let\oldItem\item
    \def\item{\addtocounter{enumi}{-2}\oldItem}
    \begin{enumerate}[#2] \itemsep3pt
    \setcounter{enumi}{#1}
    \addtocounter{enumi}{1}}
  {\end{enumerate}}

\newenvironment{education}[4]
{%\filbreak
\item
  \begin{tabular*}{7.5in}{l@{\extracolsep{\fill}}r}
    \textbf{#1} & \textit{#2} \\
		#3 & \small{#4} \\
  \end{tabular*}
  % \begin{itemize} \setlength{\parskip}{-1pt}\item
  }
  { % \end{itemize}
}

\newcommand{\service}[1]{\item $\bullet$ \hspace{1ex}\parbox{7.2in}{#1}}

\newenvironment{position}[5]
{%\filbreak
\item
  \begin{tabular*}{7.5in}{l@{\extracolsep{\fill}}r}
    \textbf{#1} & \textit{#2} \\
		\hspace{1ex} #3 & \small{#4} \\
  \end{tabular*}
\item \hspace{1ex} \parbox{7.3in}{\hspace{4ex}#5}
  % \begin{itemize} \setlength{\parskip}{-1pt}\item
  }
  { % \end{itemize}
}

\newenvironment{publication}[5]
{ \item
  \begin{tabular*}{7.5in}{p{6.3in}@{\extracolsep{\fill}}r}
    \href{#1}{\textbf{#2}} & \textit{#3}\\ #4 &\\ \textit{#5}&\\
  \end{tabular*}
} {}

\newenvironment{publicationNote}[6]
{ \item
  \begin{tabular*}{7.5in}{p{6.3in}@{\extracolsep{\fill}}r}
    \href{#1}{\textbf{#2}} & \textit{#3}\\ #4 &\\ \textit{#5}\\ \textbf{#6} \\
  \end{tabular*}
} {}

\newenvironment{talk}[4]
{ \item
  \begin{tabular*}{7.5in}{l@{\extracolsep{\fill}}r}
    \textbf{#1} & \textit{#2} \\
		\hspace{1ex} #3 & \small{#4}
\end{tabular*}
} {}

\newcommand{\reference}[7]{%
  \begin{tabular}{@{}l@{}}%
    \textbf{#1}\\%
    #2\\%
    #3\\%
    #4\\%
    #5\\%
    #6\\%
    \href{mailto:#7}{#7}%
  \end{tabular}%
}

\newcommand{\referencesmall}[6]{%
  \begin{tabular}{@{}l@{}}%
    \textbf{#1}\\%
    #2\\%
    #3\\%
    #4\\%
    #5\\%
    \href{mailto:#6}{#6}%
  \end{tabular}%
}


\newcommand{\refrow}[2]{%
\item \begin{tabular*}{0.9\textwidth}{@{}p{3.0in}@{\hspace*{0.3in}}p{3.0in}@{}} #1 & #2 \end{tabular*}
\vspace*{1ex}}

\newenvironment{region}[3]{%
  \vspace*{0.5ex}
  {\scalebox{1.4}{\textbf{#1}}}
  \begin{benumerate}{#3}{\color{RoyalBlue}#2}}
  {\end{benumerate}\vspace{0.8ex}}

\newenvironment{nonumregion}[1]{%
\begin{region}{#1}{}{1}}
{\end{region}}

\newenvironment{itemregion}[1]{
	\vspace*{0.5ex}
	{\scalebox{1.4}{\textbf{#1}}}
	\begin{itemize}\itemsep1pt}
	{\end{itemize}\vspace{0.8ex}}

\newcommand{\oral}{{\color{PineGreen}\textsc{oral}}}
\newcommand{\poster}{{\color{BurntOrange}\textsc{poster}}}

\color{darkgrey}
\begin{document}

\begin{tikzpicture}[remember picture,overlay]
  \hypersetup{urlcolor=zest}
  \node [rectangle, fill=grey, anchor=north, minimum width=\paperwidth, minimum height=4cm] (box) at (current page.north){};
  \node [anchor=center] (name) at (box) {%
    \color{white}%
    \begin{tabular*}{7.5in}{@{\extracolsep{\fill}}lr}
      \\
      \textbf{\huge{Curriculum Vitae}} & \today\\*[2ex]
      \textbf{\large{KC Sivaramakrishnan}} & {\url{http://kcsrk.info}}\\[0.5ex]
      Computer Laboratory &  15 JJ Thomson Avenue\\
      University of Cambridge & Cambridge CB3 0FD \\
      \href{mailto:sk826@cl.cam.ac.uk}{sk826@cl.cam.ac.uk}
      & \emph{Tel}: \textrm{+44 79828 42499}\\
    \end{tabular*}
  };
\end{tikzpicture}
\vspace{0.9in}

\hypersetup{urlcolor=cyan}

\begin{nonumregion}{\lbar{Mahogany}Summary}
\item \hspace{1ex} \parbox{7.3in}{I am interested in the design and implementation of
concurrent programming languages targeting scalable platforms such as many-core
processors and compute clouds. My research spans programming models, compilers,
static analysis, schedulers, threading systems, and memory management.}
\end{nonumregion}

\begin{nonumregion}{\lbar{red}Education}
  \begin{education}{PhD --- Computer Science}
    {May 2011 -- Dec 2014}
    {}
    {Purdue University, USA}
    \vspace{-3ex}
		\item
			\begin{itemize}
			\item Thesis Title: \href{http://kcsrk.info/papers/dissertation_dec14.pdf}{\textbf{Functional Programming Abstractions for Weakly Consistent Systems}}
				\vspace{-0.5ex}
			\item Advisor: Suresh Jagannathan
			\end{itemize}
  \end{education}
  \begin{education} {Master of Science --- Computer Science}
    {Aug 2008 -- May 2011}
    {}
    {Purdue University, USA}
		\vspace*{-3ex}
		\item \qquad GPA: 3.94/4
  \end{education}

\item
  \begin{tabular*}{7.5in}{l@{\extracolsep{\fill}}r}
    \textbf{Bachelor of Engineering --- Computer Science and Engineering} & \textit{Aug 2004 -- May 2008} \\
		 & \small{PSG College of Technology} \\
		& \small{Anna University, India} \\
  \end{tabular*}
		\vspace*{-5.5ex}
		\item \qquad GPA: 9.55/10

\end{nonumregion}

\begin{nonumregion} {\lbar{orange}Experience}
	\begin{position}{Research Associate, University of Cambridge}{Dec 2014 -- present}{Advisor: Anil Madhavapeddy}{Cambridge, UK}
		 {Developing \href{https://github.com/ocamllabs/ocaml-multicore}{Multicore OCaml} under the OCaml Labs initiative in the Computer Laboratory.}
	\end{position}

	\begin{position}{Research Assistant, Purdue University}{Aug 2008 -- Dec 2014}{Advisor: Suresh Jagannathan}{West Lafayette, IN, USA}
		 {My research focused on discovering new language abstractions and
		 developing runtime system techniques to ease programming weakly consistent
		 systems. To this end, I have built
		 \href{http://multimlton.cs.purdue.edu}{MultiMLton}, a parallel and
		 distributed extension of MLton Standard ML compiler and runtime and
		 \href{http://kcsrk.info/Quelea}{Quelea}, a shallow extension of Haskell
		 for declarative programming over eventually consistent data stores.}
	\end{position}

	\item \begin{tabular*}{7.5in}{l@{\extracolsep{\fill}}r}
		\textbf{Teaching Assistant, Purdue University} & \small{West Lafayette, IN, USA}\\
		\hspace{1ex} Undergraduate C Programming (CS180) & \textit{Aug 2012 -- Dec 2012} \\
		\hspace{1ex} Graduate Programming Languages (CS565) & \textit{Aug 2011 -- Dec 2011} \\
		\end{tabular*}
	\item \hspace{1ex} \parbox{7.3in}{\hspace{4ex}My tasks included designing and
	evaluating weekly projects, office hours for one-on-one instruction, and
	grading.}

	\begin{position}{Research Intern, Microsoft Research, Cambridge}{Feb 2012 -- May 2012}
		{Advisors: Tim Harris, Simon Marlow, and Simon Peyton Jones}{Cambridge, UK}
		{I developed a concurrency substrate for Glasgow Haskell Compiler (GHC) to
		allow programmers to modularly implement user-level schedulers and
		concurrency libraries for Haskell threads in Haskell, without having to
		re-engineer critical runtime system components. The concurrency substrate
		is built around one-shot continuations and uses transactional memory for
		coordination.}
	\end{position}

	\begin{position}{Research Intern, Samsung Information Systems America (R\&D)}{May 2010 -- Aug 2010}
		{Advisor: Daniel Waddington}{San Jose, CA, USA}
		{I was part of the core team that developed SNAPPLE programming language --
		a safe and concurrent extension of C++ targeted at many-core processors.
		The task involved designing language extensions for concurrency, compiler
		extensions for safety, and a runtime for executing large number of
		lightweight threads. SNAPPLE was implemented as a veneer on top of C++
		using LLNL Rose source-to-source compiler.}
	\end{position}

	\begin{position}{Intern, Advanced Numerical Research and Analysis Group}{Dec 2007 -- Apr 2008}
		{Advisor: Sankar Chnab}{Hyderabad, India}
		{As a part of the Compiler Engineering group, I ported Kaffe, an open
		source Java VM to an embedded microprocessor ANUPAMA and a desktop
		processor ABACUS. Developed a lightweight threading subsystem, and
		implemented a JIT backed for ABACUS.}
	\end{position}
\end{nonumregion}

\begin{region} {\lbar{purple}Journal Publications}{{J}1}{2}

	\begin{publication}{http://kcsrk.info/papers/schedact_jfp15.pdf}
		{Composable Scheduler Activations for Haskell}
		{Nov 2015}{KC Sivaramakrishnan, Tim Harris, Simon Marlow, Simon Peyton Jones}
		{Accepted with minor revisions for Journal of Functional Programming (JFP)}
	\end{publication}

	\begin{publication} {http://kcsrk.info/papers/multimlton_jfp14.pdf}
		{MultiMLton: A Multicore-aware Runtime for Standard ML}
		{Nov 2014} {KC Sivaramakrishnan, Lukasz Ziarek, Suresh Jagannathan}
		{Journal of Functional Programming (JFP), 24(6): 613 -- 674}
	\end{publication}

	\begin{publicationNote}{http://kcsrk.info/papers/sting_scp13.pdf}{Efficient Sessions}
		{Feb 2013}{KC Sivaramakrishnan, Mohammad Qudeisat, Lukasz Ziarek, Karthik Nagaraj, Patrick Eugster}
		{Science of Computer Programming (SCP), 78(2): 147 -- 167}
		{Invited paper}
	\end{publicationNote}
\end{region}

\begin{region} {\lbar{purple}Conference Publications}{{C}1}{7}
	\begin{publication} {http://kcsrk.info/papers/quelea_pldi15.pdf}
		{Declarative Programming over Eventually Consistent Data Stores}
		{Jun 2015} {KC Sivaramakrishnan, Gowtham Kaki, Suresh Jagannathan}
		{International Conference on Programming Language Design and Implementation (PLDI)}
	\end{publication}

	\begin{publication}{http://kcsrk.info/papers/rxcml_padl14.pdf}
		{Rx-CML: A Prescription for Safely Relaxing Synchrony}
		{Jan 2014}{KC Sivaramakrishnan, Lukasz Ziarek, Suresh Jagannathan}
		{Symposium on Practical Aspects of Declarative Languages (PADL)}
	\end{publication}

	\begin{publicationNote}{http://kcsrk.info/papers/mmscc_marc12.pdf}
		{A Coherent and Managed Runtime for ML on the SCC}
		{Nov 2012}{KC Sivaramakrishnan, Lukasz Ziarek, Suresh Jagannathan}
		{Many-core Architecture Research Community Symposium (MARC)}
		{Best paper award}
	\end{publicationNote}

	\begin{publication}{http://kcsrk.info/papers/mmgc_ismm12.pdf}
		{Eliminating Read Barriers through Procrastination and Cleanliness}
		{Jun 2012}{KC Sivaramakrishnan, Lukasz Ziarek, Suresh Jagannathan}
		{International Symposium on Memory Management (ISMM)}
	\end{publication}

	\begin{publication}{http://kcsrk.info/papers/acml_pldi11.pdf}
		{Composable Asynchronous Events}
		{Jun 2011}{Lukasz Ziarek, KC Sivaramakrishnan, Suresh Jagannathan}
		{International Conference on Programming Language Design and Implementation (PLDI)}
	\end{publication}

	\begin{publication}{http://kcsrk.info/papers/sting_coordination10.pdf}
		{Efficient Session Type Guided Distributed Interaction}
		{June 2010}{KC Sivaramakrishnan, Karthik Nagaraj, Lukasz Ziarek, Patrick Eugster}
		{International Conference on Coordination Models and Languages (COORDINATION)}
	\end{publication}

	\begin{publication}{http://kcsrk.info/papers/memo_icfp09.pdf}
		{Partial Memoization of Concurrency and Communication}
		{Sep 2009}{Lukasz Ziarek, KC Sivaramakrishnan, Suresh Jagannathan}
		{International Conference on Functional Programming (ICFP)}
	\end{publication}
\end{region}


\begin{region} {\lbar{purple}Workshop Publications}{{W}1}{4}

	\begin{publication}{http://kcsrk.info/papers/effects_ocaml15.pdf}
		{Effective Concurrency with Algebraic Effects}
		{Sep 2015}{Stephen Dolan, Leo White, KC Sivaramakrishnan, Jeremy Yallop and Anil Madhavapeddy}
		{OCaml Workshop}
	\end{publication}

	\begin{publication}{http://kcsrk.info/papers/mmcloud_mlw13.pdf}
		{Migrating MultiMLton to the Cloud}
		{Sep 2013}{KC Sivaramakrishnan, Lukasz Ziarek, Suresh Jagannathan}
		{ML Workshop}
	\end{publication}

	\begin{publication}{http://kcsrk.info/papers/snapple_sfma11.pdf}
		{Scalable Lightweight Task Management Schemes for MIMD Processors}
		{Apr 2011}{Daniel G. Waddington, Chen Tian, KC Sivaramakrishnan}
		{Workshop on Systems for Future Multi-Core Architectures (SFMA)}
	\end{publication}

	\begin{publication}{http://kcsrk.info/papers/multimlton_mlw10.pdf}
		{The Design Rationale for MultiMLton}
		{Sep 2010}{Suresh Jagannathan, Armand Navabi, KC Sivaramakrishnan, Lukasz Ziarek}
		{ML Workshop}
	\end{publication}

	\begin{publication}{http://kcsrk.info/papers/parasites_damp10.pdf}
		{Lightweight Asynchrony using Parasitic Threads}
		{Jan 2010}{KC Sivaramakrishnan, Lukasz Ziarek, Raghavendra Prasad, Suresh Jagannathan}
		{Workshop on Declarative Aspects of Multicore Programming (DAMP)}
	\end{publication}
\end{region}

\begin{region} {\lbar{purple}Technical Reports and Drafts}{{T}1}{2}

	\begin{publication}{http://kcsrk.info/papers/parasites_tech11.pdf}
		{Featherweight Threads for Communication}
		{Nov 2011}{KC Sivaramakrishnan, Lukasz Ziarek, Suresh Jagannathan}
		{Purdue University Computer Science Technical Report -- TR-11-018}
	\end{publication}

\end{region}

\begin{itemregion}{\lbar{red}Teaching/Advising}
	\service{Guest Lectures:}
	\begin{itemize}
		\service{Arrows, Advanced Functional Programming, University of Cambridge, Lent 2015--16.}
		\service{Debugging, Programming in C and C++, University of Cambridge, Michelmas 2015--16.}
	\end{itemize}
	\service{Supervisions at University of Cambridge:}
	\begin{itemize}
		\service{Algorithms, Lent 2015--16.}
		\service{Concurrent and Distributed Systems, Lent 2015--16.}
		\service{Concurrent and Distributed Systems, Michaelmas 2015--16.}
		\service{Object-oriented Programming, Michaelmas 2015--16.}
	\end{itemize}
	\service{Teaching assistantships at Purdue University}
	\begin{itemize}
		\service{Undergraduate C Programming (CS180), Aug 2012 -- Dec 2012.}
		\service{Graduate Programming Languages (CS565), Aug 2011 -- Dec 2011.}
	\end{itemize}
	\service{Projects supervised:}
	\begin{itemize}
		\service{James Wright, University of Cambridge, Mechanized semantics of Algebraic Effects in OCaml, Michelmas 2015 -- present.}
		\service{Armael Gueneau, ENS de Lyon, Algebraic Effects for js\_of\_ocaml, Sep 2015 -- present.}
		\service{Theo Laurent, ENS de Lyon, Reagents for Multicore OCaml, May 2015 -- Aug 2015.}
		\service{Guillain Potron, ENS de Lyon, Semantics of Irmin branch-consistent data store, March 2015 -- Aug 2015.}
	\end{itemize}
\end{itemregion}

\begin{itemregion}{\lbar{blue}Talks}
	\begin{talk} {OCaml Platform: Update}
			{Nov 2015}{OCaml Consortium Meeting}{Paris, France}
	\end{talk}

	\begin{talk} {Multicore OCaml: Update}
			{Nov 2015}{OCaml Developer's Meeting}{Paris, France}
	\end{talk}

	\begin{talk} {Effective Concurrency with Algebraic Effects}
			{Sep 2015}{OCaml Workshop 2015}{Vancouver, Canada}
	\end{talk}
	\begin{talk} {Functional Programming Abstractions for Weakly Consistent Systems}
	   {Dec 2014}{PhD Defense}{Purdue University}
	\end{talk}

	\begin{talk} {Functional Abstractions for Practical and Scalable Concurrent Programming}
		 {Mar 2014}{Invited Lecture}{Microsoft Research, Cambridge, UK}
	\end{talk}

	\begin{talk}{Rx-CML: A Prescription for Safely Relaxing Synchrony}
		{Jan 2014}{PADL 2014}{San Diego, CA}
	\end{talk}

	\begin{talk}{Migrating MultiMLton to the Cloud}
		{Sep 2013}{ML Workshop 2013}{Boston, MA}
	\end{talk}

	\begin{talk}{A Coherent and Managed Runtime for ML on the SCC}
		{Nov 2012}{MARC 2012}{RWTH Aachen}
	\end{talk}

	\item \begin{tabular*}{7.5in}{l@{\extracolsep{\fill}}r}
					\textbf{Eliminating Read Barriers through Procrastination and Cleanliness} \\
					\hspace{1ex} ISMM 2012, Beijing 					 	& \textit{Jun 2012} \\
					\hspace{1ex} Wrestling Wednesdays, Microsoft Research, Cambridge 	& \textit{May 2012}
				\end{tabular*}

	\begin{talk}{Lightweight Concurrency in GHC}
		{May 2012}{Wrestling Wednesdays}{Microsoft Research, Cambridge}
	\end{talk}

	\begin{talk}{Efficient Session Type guided Distributed Interaction}
		{Jun 2012}{COORDINATION 2012}{CWI Amsterdam}
	\end{talk}
\end{itemregion}

\begin{itemregion} {\lbar{blue}Professional Service}
	\service{Program Committee member: SPLASH-MARC symposium, 2013.}
	\service{Artifact Evaluation Committee member: PLDI 2015, PPoPP/CGO 2016.} \vspace{0.5ex}
	\service{Reviewer: POPL, ICFP, ASPLOS, TLDI, Concurrency and Computation:
	Practice and Experience, Software: Practice and Experience.}
\end{itemregion}

\begin{itemregion} {\lbar{blue}Awards and Recognitions}
  \service{Research Fellowship, Royal Commission for the Exhibition of 1851, 2015--2018, \pounds102,000.}
  \service{Research Fellowship, Darwin College, Cambridge, 2015--2018, \pounds900.}
	\service{Maurice H. Halstead Memorial Award for outstanding research in Software Engineering, Purdue University, 2014, \$4,000.}
	\service{Best paper award at Many-core Architecture Research Symposium at RWTH-Aachen, 2012, \$1,000.}
	\service{Invited paper in Science of Computer Programming, Vol. 78, Iss. 2 (Feb 2013).}
	\service{Glasgow Haskell Compiler (GHC) Committer.}
	\service{SIGPLAN PAC travel grant for PLDI 2012 and POPL 2014, ~\$1,500 each.}
	\service{NSF travel grant for ICFP 2013, ~\$2,000.}
\end{itemregion}

\ifdefined\genrefs

\begin{nonumregion}{\lbar{green}References}
  \refrow{
    \reference{Suresh Jagannathan}
    {Professor}
    {Department of Computer Science}
    {Purdue University}
    {305 N University St}
    {West Lafayette, IN 47906, USA}
    {suresh@cs.purdue.edu}
  }{
    \reference{Anil Madhavapeddy}
    {University Lecturer}
    {Computer Laboratory}
    {University of Cambridge}
    {15 JJ Thomson Av}
    {Cambridge, CB3 0FD, UK}
    {anil.madhavapeddy@cl.cam.ac.uk}
  }
  \refrow{
    \reference{Jan Vitek}
    {Professor of Computer Science}
    {College of Computer \& Information Science}
    {Northeastern University}
    {440 Huntinton Av}
    {Boston, MA 02115, USA}
    {j.vitek@neu.edu}
  }{
    \reference{Simon Peyton Jones}
    {Principal Researcher}
		{Programming Principles and Tools}
    {Microsoft Research Ltd}
    {21 Station Rd}
    {Cambridge CB1 2FB, UK}
    {simonpj@microsoft.com}
  }
\end{nonumregion}

\fi

\end{document}

%  LocalWords:  lr Siddharth Narayanaswamy Compositionality Advisor Siskind Yu
%  LocalWords:  Muniyandi Manivannan IIT Manya Sep Unseeability Unseeable ACS
%  LocalWords:  Barbu Michaux Compositional JAIR IEEE PAMI Cao CVPR RoyalBlue
%  LocalWords:  Burchill Coroian Fidler Mussman Salvi Shangguan Waggoner Zhang
%  LocalWords:  Uncertainity UAI ICRA VR Laparoscopic Devasahayam MMVR Xiong
%  LocalWords:  Corso Fellbaum elie Malaia Pearlmutter Talavage IPAB JT
%  LocalWords:  Colloquia AAAI ICDL PRL InnerVision Cohn Jos RUBIC Heavilon
